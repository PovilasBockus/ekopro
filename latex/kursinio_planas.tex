\documentclass[12pt,a4paper]{article}
\usepackage[utf8]{inputenc}
\usepackage[english,lithuanian]{babel}
\usepackage[L7x]{fontenc}
\usepackage{lmodern}
\usepackage{amsmath}
\usepackage{amssymb}
\usepackage{theorem}
\usepackage{bm}
%\usepackage[unicode]{hyperref}
%\usepackage{ucshyper}
\pagestyle{plain}

\newcommand{\eps}{\varepsilon}
\newcommand{\E}{\mathbf{E}}
\newcommand{\PP}{\mathbf{P}}
\theoremstyle{change}\newtheorem{salyga}{Uždavinys}

\DeclareMathOperator{\spp}{sp}

\DeclareMathOperator{\Corr}{Corr}
\topmargin=0cm
\textheight=700pt
\textwidth=430pt
\oddsidemargin=0pt
\headsep=0pt
\headheight=0pt
%\voffset=-1in
\def\qed{\relax\ifmmode\hskip2em \Box\else\unskip\nobreak\hskip1em $\Box$\fi}

\begin{document}
\begin{titlepage}
\centerline{ \large VILNIAUS UNIVERSITETAS}
\bigskip
\centerline{\large MATEMATIKOS IR INFORMATIKOS FAKULTETAS}
\smallskip

\centerline{\large  EKONOMETRINĖS ANALIZĖS KATEDRA}
\vskip 200pt
\centerline{ \large Alina \textsc{Rauktytė} ir Povilas \textsc{Bočkus}}
\vskip 50pt
\centerline{\Large Ekonometrinio projekto}
\vskip 25pt
\centerline{\bf \Large \textsc{Darbo užmokesčio nustatymo mechanizmas}}
\vskip 25pt
\centerline{\Large PLANAS}
\bigskip
\vskip 50pt
\begin{flushright}
 Kursinio projekto vadovas: 
 doc. Remigijus Lapinskas
\end{flushright}
\hfill Ekonometrija, III kursas, 2 grupė
\vskip 150pt
\centerline{\large VILNIUS 2011}
\end{titlepage}

\begin{center}
\textbf{\Large\textsc{Ekonometrinio projekto planas}}
\end{center}
\vskip 15pt
\begin{enumerate}



\item  \underline{\textbf{{Informacijos duota tema rinkimas}}}
\vskip 10pt

Medžiagos apie darbo užmokesčio nustatymo mechanizmo rinkimas, gilinimasis į temą. Paieška turėtų būti vykdoma ekonominio pobūdžio literatūroje: knygose, makroekonomikos vadovėliuose, internete. Medžiagos susisteminimas. Atsirinkti tai, kas bus reikalinga.

\item  \underline{\textbf{{Duomenų rinkimas ir analizė}}}
\vskip 10pt

Antrajame darbo etape būtina pasirinkti reikalingus kintamuosius. Tai padarius, ieškomi atitinkami duomenys. Pasirenkama duomenų apimtis ir laiko periodas. Atliekama pirminė statistinė duomenų analizė. 


\item  \underline{\textbf{{Ekonometrinio modelio specifikacija}}}
\vskip 10pt

Atliekami reikalingi veiksmai prieš modelio specifikaciją, pvz.: kointegravimo sąryšių nustatymas. Remiantis ekonominėmis prielaidomis, ir turimomis žiniomis, specifikuojame modelį. 

\item  \underline{\textbf{{Ekonometrinio modelio parametrai}}}
\vskip 10pt

Šiuo etapu įvertiname pasirinkto modelio parametrus. Tikėtina, kad bus  keli modelio variantai, su skirtingais kintamaisiais ir parametrais. Ištirsime kiekvieno modelio tikslumą. 

\item  \underline{\textbf{{Geriausio modelio parinkimas}}}
\vskip 10pt

Atsižvelgiant į įvairius kriterijus pasirinksime geriausią modelį, kuris tiksliausiai atitinka turimus duomenis. 

\item  \underline{\textbf{{Išvados}}}
\vskip 10pt

Šiame etape būtina atlikti gautų rezultatų analizę. Rezultatus susieti su ekonominėmis išvadomis, bei interpretacijomis.

\item  \underline{\textbf{{Kursinio darbo realizacija}}}
\vskip 10pt

Numatomas paskutinis etapas, kurio metu visas atliktas darbas užbaigiamas atsižvelgiant į kursinio darbo rašymo reikalavimus. Pasiruošimas darbo pristatymui ir atsiskaitymui.


\end{enumerate}



     
     
\end{document}